%%% ===============chapter 6 starts here ============================

\chapter{GENERAL CONCLUSIONS}\label{ch:conclusion}

\section{General discussion} 

Visual statistical inference offers promise in situations when there is no formal way of testing hypothesis. In Chapter \ref{ch:largepsmalln} visual inference procedures were applied to classification techniques in high dimension, low sample size (HDLSS) data. The results suggest that visual inference may be effective for improving the understanding of the emptiness of space in this type of data. With visual inference it was seen that people can visually detect real separation as different from noise up to a reasonably high dimension, for 1D and 2D projections. Visual inference provides a calibration for reading the separation. Although it was not discussed in Chapter \ref{ch:largepsmalln}, we also learned from visual inference that the projection pursuit optimization procedure in \texttt{tourr} package is performing correctly. It is possible that visual inference might be used to calibrate results of similar algorithms, where the optimization is used to yield visual products, like multidimensional scaling, PCA, independent component analysis (ICA) and local linear embeddings. 

In Chapter \ref{ch:metrics}, we discussed a possible shortcoming of lineup protocol. Unlike classical inference, human subjects have to base their responses on a finite set of null plots although the sampling distribution of null plots is theoretically infinite. Distance metrics were introduced to measure the quality of a lineup and compared to the response of the human subjects. The results suggest that the distance metrics work well in specific situations. The distance metrics considering plot designs works better than generic distance metrics. These metrics may help in designing future Amazon Turk Experiments. Based on the quality of a lineup, more subjects may be allotted to difficult lineups. Comparing different distance metrics on the same lineup may reveal the reason behind the choice made by human subjects. We believe our work with the distance metrics will make lineup protocol more powerful and will help nullifying the effects of finite null plots. 

The R package in Chapter \ref{ch:nullabor} provides open source tools to use visual inference and distance metrics. The package provides ways to generate lineup plots automatically for different null generating mechanisms. It also provides methods to measure the quality of a lineup based on different distance metrics and diagnostic plots to compare several lineups. The users may also use their customized distance metrics to compare lineups. By making this freely available we hope  to see increased application of visual inference in situations where there are no conventional testing methods.

\section{Possibilities for future research}

This dissertation provides the ground work for applications of visual inference. There are several natural next steps for this research. 

\begin{enumerate} 
\item Visual inference can be extended to produce confidence intervals. For example, in HDLSS problems to generate a range for the ratio of sample size to dimension at which separation between groups is indistinguishable from noise? In my initial experiment, it was clear that at some point real separation was indistinguishable from noise, and it would be interesting to compare how people determine where this occurs with what has been done analytically on these problems.  
\item Visual statistical inference may improve learning in the statistics curriculum. An initial study was done with my undergraduate class and our experimental data suggest that visual inference was very intuitive decision making process which does not require advanced knowledge on mathematics or statistics. It may even be possible to incorporate into elementary school curriculum to build early understanding of randomness. 
%\item Gene expression analysis is another situation where visual inference can be widely used. This is a classical case of high dimension, low sample size data. Plotting the most significant genes from such datasets do not reveal obvious structures even after adjustments with false discovery rate (FDR). A lineup with the significant gene as the actual plot and significant genes obtained from permuted design as null plots may reveal different results.
%\item  Distance metrics may be used for plot designs. The results for distance metrics are compared with the response of the human subjects. If a particular distance metric shows promising results, a plot can be designed based on the particular characteristics of the distance metric so that they can be easily identified by the human subjects.
\end{enumerate}